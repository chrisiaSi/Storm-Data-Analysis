% Options for packages loaded elsewhere
\PassOptionsToPackage{unicode}{hyperref}
\PassOptionsToPackage{hyphens}{url}
%
\documentclass[
]{article}
\usepackage{amsmath,amssymb}
\usepackage{iftex}
\ifPDFTeX
  \usepackage[T1]{fontenc}
  \usepackage[utf8]{inputenc}
  \usepackage{textcomp} % provide euro and other symbols
\else % if luatex or xetex
  \usepackage{unicode-math} % this also loads fontspec
  \defaultfontfeatures{Scale=MatchLowercase}
  \defaultfontfeatures[\rmfamily]{Ligatures=TeX,Scale=1}
\fi
\usepackage{lmodern}
\ifPDFTeX\else
  % xetex/luatex font selection
\fi
% Use upquote if available, for straight quotes in verbatim environments
\IfFileExists{upquote.sty}{\usepackage{upquote}}{}
\IfFileExists{microtype.sty}{% use microtype if available
  \usepackage[]{microtype}
  \UseMicrotypeSet[protrusion]{basicmath} % disable protrusion for tt fonts
}{}
\makeatletter
\@ifundefined{KOMAClassName}{% if non-KOMA class
  \IfFileExists{parskip.sty}{%
    \usepackage{parskip}
  }{% else
    \setlength{\parindent}{0pt}
    \setlength{\parskip}{6pt plus 2pt minus 1pt}}
}{% if KOMA class
  \KOMAoptions{parskip=half}}
\makeatother
\usepackage{xcolor}
\usepackage[margin=1in]{geometry}
\usepackage{color}
\usepackage{fancyvrb}
\newcommand{\VerbBar}{|}
\newcommand{\VERB}{\Verb[commandchars=\\\{\}]}
\DefineVerbatimEnvironment{Highlighting}{Verbatim}{commandchars=\\\{\}}
% Add ',fontsize=\small' for more characters per line
\usepackage{framed}
\definecolor{shadecolor}{RGB}{248,248,248}
\newenvironment{Shaded}{\begin{snugshade}}{\end{snugshade}}
\newcommand{\AlertTok}[1]{\textcolor[rgb]{0.94,0.16,0.16}{#1}}
\newcommand{\AnnotationTok}[1]{\textcolor[rgb]{0.56,0.35,0.01}{\textbf{\textit{#1}}}}
\newcommand{\AttributeTok}[1]{\textcolor[rgb]{0.13,0.29,0.53}{#1}}
\newcommand{\BaseNTok}[1]{\textcolor[rgb]{0.00,0.00,0.81}{#1}}
\newcommand{\BuiltInTok}[1]{#1}
\newcommand{\CharTok}[1]{\textcolor[rgb]{0.31,0.60,0.02}{#1}}
\newcommand{\CommentTok}[1]{\textcolor[rgb]{0.56,0.35,0.01}{\textit{#1}}}
\newcommand{\CommentVarTok}[1]{\textcolor[rgb]{0.56,0.35,0.01}{\textbf{\textit{#1}}}}
\newcommand{\ConstantTok}[1]{\textcolor[rgb]{0.56,0.35,0.01}{#1}}
\newcommand{\ControlFlowTok}[1]{\textcolor[rgb]{0.13,0.29,0.53}{\textbf{#1}}}
\newcommand{\DataTypeTok}[1]{\textcolor[rgb]{0.13,0.29,0.53}{#1}}
\newcommand{\DecValTok}[1]{\textcolor[rgb]{0.00,0.00,0.81}{#1}}
\newcommand{\DocumentationTok}[1]{\textcolor[rgb]{0.56,0.35,0.01}{\textbf{\textit{#1}}}}
\newcommand{\ErrorTok}[1]{\textcolor[rgb]{0.64,0.00,0.00}{\textbf{#1}}}
\newcommand{\ExtensionTok}[1]{#1}
\newcommand{\FloatTok}[1]{\textcolor[rgb]{0.00,0.00,0.81}{#1}}
\newcommand{\FunctionTok}[1]{\textcolor[rgb]{0.13,0.29,0.53}{\textbf{#1}}}
\newcommand{\ImportTok}[1]{#1}
\newcommand{\InformationTok}[1]{\textcolor[rgb]{0.56,0.35,0.01}{\textbf{\textit{#1}}}}
\newcommand{\KeywordTok}[1]{\textcolor[rgb]{0.13,0.29,0.53}{\textbf{#1}}}
\newcommand{\NormalTok}[1]{#1}
\newcommand{\OperatorTok}[1]{\textcolor[rgb]{0.81,0.36,0.00}{\textbf{#1}}}
\newcommand{\OtherTok}[1]{\textcolor[rgb]{0.56,0.35,0.01}{#1}}
\newcommand{\PreprocessorTok}[1]{\textcolor[rgb]{0.56,0.35,0.01}{\textit{#1}}}
\newcommand{\RegionMarkerTok}[1]{#1}
\newcommand{\SpecialCharTok}[1]{\textcolor[rgb]{0.81,0.36,0.00}{\textbf{#1}}}
\newcommand{\SpecialStringTok}[1]{\textcolor[rgb]{0.31,0.60,0.02}{#1}}
\newcommand{\StringTok}[1]{\textcolor[rgb]{0.31,0.60,0.02}{#1}}
\newcommand{\VariableTok}[1]{\textcolor[rgb]{0.00,0.00,0.00}{#1}}
\newcommand{\VerbatimStringTok}[1]{\textcolor[rgb]{0.31,0.60,0.02}{#1}}
\newcommand{\WarningTok}[1]{\textcolor[rgb]{0.56,0.35,0.01}{\textbf{\textit{#1}}}}
\usepackage{graphicx}
\makeatletter
\def\maxwidth{\ifdim\Gin@nat@width>\linewidth\linewidth\else\Gin@nat@width\fi}
\def\maxheight{\ifdim\Gin@nat@height>\textheight\textheight\else\Gin@nat@height\fi}
\makeatother
% Scale images if necessary, so that they will not overflow the page
% margins by default, and it is still possible to overwrite the defaults
% using explicit options in \includegraphics[width, height, ...]{}
\setkeys{Gin}{width=\maxwidth,height=\maxheight,keepaspectratio}
% Set default figure placement to htbp
\makeatletter
\def\fps@figure{htbp}
\makeatother
\setlength{\emergencystretch}{3em} % prevent overfull lines
\providecommand{\tightlist}{%
  \setlength{\itemsep}{0pt}\setlength{\parskip}{0pt}}
\setcounter{secnumdepth}{-\maxdimen} % remove section numbering
\ifLuaTeX
  \usepackage{selnolig}  % disable illegal ligatures
\fi
\IfFileExists{bookmark.sty}{\usepackage{bookmark}}{\usepackage{hyperref}}
\IfFileExists{xurl.sty}{\usepackage{xurl}}{} % add URL line breaks if available
\urlstyle{same}
\hypersetup{
  pdftitle={Analysis of Severe Weather Events in the U.S. (1950-2011): Health and Economic Impacts},
  pdfauthor={Chr},
  hidelinks,
  pdfcreator={LaTeX via pandoc}}

\title{Analysis of Severe Weather Events in the U.S. (1950-2011): Health
and Economic Impacts}
\author{Chr}
\date{2024-09-14}

\begin{document}
\maketitle

Severe weather events, such as hurricanes, tornadoes, floods, and
heatwaves, can have devastating impacts on public health and the
economy. In this analysis, we explore data from the U.S. National
Oceanic and Atmospheric Administration's (NOAA) Storm Database, covering
storm events from 1950 to 2011. This database provides comprehensive
records of storms and other severe weather incidents, including
information on fatalities, injuries, and damage to properties and crops.

The goal of this analysis is to determine which types of severe weather
events have been most harmful to population health and which events have
had the greatest economic consequences across the United States. By
examining variables such as event types, fatalities, injuries, and
property damage, we aim to provide insights that can help local
governments and decision-makers prioritize resources and preparedness
efforts for future severe weather incidents. The analysis uses
visualizations and summary statistics to identify key trends and the
most impactful event types in terms of human health and economic loss.

\hypertarget{synopsis}{%
\subsection{Synopsis}\label{synopsis}}

This report explores the U.S. National Oceanic and Atmospheric
Administration's (NOAA) storm database, aiming to identify which types
of severe weather events have the greatest impact on public health and
economic damages across the United States. The data spans from 1950 to
2011 and tracks key weather-related incidents, including fatalities,
injuries, and property damage. The analysis focuses on two primary
questions: which types of events are most harmful to population health
and which types of events have the greatest economic consequences. The
analysis includes data loading, processing, and visualization of the
results through summary tables and figures. To answer these questions,
we first process the raw data by cleaning and transforming the necessary
variables.

The data are analyzed by aggregating the number of fatalities, injuries,
property damage, and crop damage for each type of weather event
(EVTYPE). Our findings show that tornadoes have caused the highest
number of fatalities and injuries, making them the most dangerous
weather event for public health. In terms of economic impact, floods,
hurricanes, and tornadoes have caused the greatest financial damage to
property and crops.

Overall, the analysis highlights tornadoes as the most harmful event in
terms of human health, while floods and hurricanes lead in economic
damage. These insights can inform future disaster preparedness and
resource allocation to mitigate the effects of severe weather events.

\hypertarget{data-processing}{%
\subsection{Data Processing}\label{data-processing}}

\hypertarget{data-loading}{%
\subsubsection{Data Loading}\label{data-loading}}

The data is provided in a compressed CSV format, so the first step
involves loading it into R for analysis.

\begin{Shaded}
\begin{Highlighting}[]
\NormalTok{storm\_data }\OtherTok{\textless{}{-}} \FunctionTok{read.csv}\NormalTok{((}\StringTok{"repdata\_data\_StormData.csv.bz2"}\NormalTok{))}
\end{Highlighting}
\end{Shaded}

\hypertarget{data-cleaning}{%
\subsubsection{Data Cleaning}\label{data-cleaning}}

Data cleaning of weather event types as there are several duplicate
entries with minor differences pointing to the same weather event.

\begin{Shaded}
\begin{Highlighting}[]
\CommentTok{\#985 unique entries}
\FunctionTok{length}\NormalTok{(}\FunctionTok{unique}\NormalTok{(storm\_data}\SpecialCharTok{$}\NormalTok{EVTYPE))}
\end{Highlighting}
\end{Shaded}

\begin{verbatim}
## [1] 985
\end{verbatim}

\begin{Shaded}
\begin{Highlighting}[]
\CommentTok{\#Converting all the weather events to lower case}
\NormalTok{evtype }\OtherTok{\textless{}{-}} \FunctionTok{trimws}\NormalTok{(}\FunctionTok{tolower}\NormalTok{(storm\_data}\SpecialCharTok{$}\NormalTok{EVTYPE))}

\CommentTok{\#Removing special characters from them except for dot/decimal character.}
\NormalTok{evtype }\OtherTok{\textless{}{-}} \FunctionTok{gsub}\NormalTok{(}\StringTok{\textquotesingle{}[\^{}a{-}zA{-}Z0{-}9. ]\textquotesingle{}}\NormalTok{,}\StringTok{\textquotesingle{}\textquotesingle{}}\NormalTok{,evtype)}

\CommentTok{\#thunderstorm wind is same as tstm wind, hence, we are using a combining them}
\NormalTok{evtype }\OtherTok{\textless{}{-}} \FunctionTok{gsub}\NormalTok{(}\StringTok{\textquotesingle{}thunderstorm\textquotesingle{}}\NormalTok{,}\StringTok{\textquotesingle{}tstm\textquotesingle{}}\NormalTok{,evtype)}

\CommentTok{\#Saving back to storm data}
\NormalTok{storm\_data}\SpecialCharTok{$}\NormalTok{EVTYPE }\OtherTok{\textless{}{-}}\NormalTok{ evtype}

\CommentTok{\#875 unique entries, 110 duplicates have been handled}
\FunctionTok{length}\NormalTok{(}\FunctionTok{unique}\NormalTok{(storm\_data}\SpecialCharTok{$}\NormalTok{EVTYPE))}
\end{Highlighting}
\end{Shaded}

\begin{verbatim}
## [1] 866
\end{verbatim}

\hypertarget{summarizing-health-impacts}{%
\subsubsection{Summarizing Health
Impacts}\label{summarizing-health-impacts}}

Across the United States, which types of events are most harmful with
respect to population health? We focus on two measures for population
health: FATALITIES and INJURIES. The goal is to aggregate these metrics
by event type (EVTYPE) to identify the most harmful weather events.

\begin{Shaded}
\begin{Highlighting}[]
\FunctionTok{library}\NormalTok{(dplyr)}
\end{Highlighting}
\end{Shaded}

\begin{verbatim}
## 
## Attaching package: 'dplyr'
\end{verbatim}

\begin{verbatim}
## The following objects are masked from 'package:stats':
## 
##     filter, lag
\end{verbatim}

\begin{verbatim}
## The following objects are masked from 'package:base':
## 
##     intersect, setdiff, setequal, union
\end{verbatim}

\begin{Shaded}
\begin{Highlighting}[]
\CommentTok{\# Aggregating fatalities and injuries by event type}
\NormalTok{health\_impact }\OtherTok{\textless{}{-}}\NormalTok{ storm\_data }\SpecialCharTok{\%\textgreater{}\%}
  \FunctionTok{group\_by}\NormalTok{(EVTYPE) }\SpecialCharTok{\%\textgreater{}\%}
  \FunctionTok{summarise}\NormalTok{(}\AttributeTok{total\_fatalities =} \FunctionTok{sum}\NormalTok{(FATALITIES, }\AttributeTok{na.rm =} \ConstantTok{TRUE}\NormalTok{),}
            \AttributeTok{total\_injuries =} \FunctionTok{sum}\NormalTok{(INJURIES, }\AttributeTok{na.rm =} \ConstantTok{TRUE}\NormalTok{)) }\SpecialCharTok{\%\textgreater{}\%}
  \FunctionTok{arrange}\NormalTok{(}\FunctionTok{desc}\NormalTok{(total\_fatalities }\SpecialCharTok{+}\NormalTok{ total\_injuries))}

\CommentTok{\# Displaying the top 10 harmful events to population health}
\NormalTok{top\_health\_events }\OtherTok{\textless{}{-}}\NormalTok{ health\_impact }\SpecialCharTok{\%\textgreater{}\%} \FunctionTok{top\_n}\NormalTok{(}\DecValTok{10}\NormalTok{, total\_fatalities }\SpecialCharTok{+}\NormalTok{ total\_injuries)}
\NormalTok{top\_health\_events}\SpecialCharTok{$}\NormalTok{EVTYPE}
\end{Highlighting}
\end{Shaded}

\begin{verbatim}
##  [1] "tornado"        "tstm wind"      "excessive heat" "flood"         
##  [5] "lightning"      "heat"           "flash flood"    "ice storm"     
##  [9] "winter storm"   "high wind"
\end{verbatim}

Tornadoes have caused the highest number of fatalities and injuries,
making them the most dangerous weather event for public health. This is
followed by Thunderstorm Wind and Excessive Heat in terms of fatalities
and injuries combined.

\hypertarget{summarizing-economic-impacts}{%
\subsubsection{Summarizing Economic
Impacts}\label{summarizing-economic-impacts}}

Across the United States, which types of events have the greatest
economic consequences? Economic impacts are captured by two variables:
PROPDMG (property damage) and CROPDMG (crop damage). We will aggregate
these metrics by event type to identify the events with the greatest
economic consequences.

\begin{Shaded}
\begin{Highlighting}[]
\CommentTok{\# Aggregating economic damages by event type}
\NormalTok{economic\_impact }\OtherTok{\textless{}{-}}\NormalTok{ storm\_data }\SpecialCharTok{\%\textgreater{}\%}
  \FunctionTok{group\_by}\NormalTok{(EVTYPE) }\SpecialCharTok{\%\textgreater{}\%}
  \FunctionTok{summarise}\NormalTok{(}\AttributeTok{total\_prop\_dmg =} \FunctionTok{sum}\NormalTok{(PROPDMG, }\AttributeTok{na.rm =} \ConstantTok{TRUE}\NormalTok{),}
            \AttributeTok{total\_crop\_dmg =} \FunctionTok{sum}\NormalTok{(CROPDMG, }\AttributeTok{na.rm =} \ConstantTok{TRUE}\NormalTok{),}
            \AttributeTok{total\_economic\_dmg =}\NormalTok{ total\_prop\_dmg }\SpecialCharTok{+}\NormalTok{ total\_crop\_dmg) }\SpecialCharTok{\%\textgreater{}\%}
  \FunctionTok{arrange}\NormalTok{(}\FunctionTok{desc}\NormalTok{(total\_economic\_dmg))}

\CommentTok{\# Displaying the top 10 events with greatest economic impact}
\NormalTok{top\_economic\_events }\OtherTok{\textless{}{-}}\NormalTok{ economic\_impact }\SpecialCharTok{\%\textgreater{}\%} \FunctionTok{top\_n}\NormalTok{(}\DecValTok{10}\NormalTok{, total\_economic\_dmg)}
\NormalTok{top\_economic\_events}\SpecialCharTok{$}\NormalTok{EVTYPE}
\end{Highlighting}
\end{Shaded}

\begin{verbatim}
##  [1] "tornado"      "tstm wind"    "flash flood"  "hail"         "flood"       
##  [6] "lightning"    "tstm winds"   "high wind"    "winter storm" "heavy snow"
\end{verbatim}

\hypertarget{results}{%
\subsection{Results}\label{results}}

\hypertarget{health-impacts-of-weather-events}{%
\subsubsection{Health Impacts of Weather
Events}\label{health-impacts-of-weather-events}}

The following figure shows the top 10 severe weather events that have
the highest combined number of fatalities and injuries across the United
States from 1950 to 2011. Tornadoes appear to be the most harmful event
in terms of population health.

\begin{Shaded}
\begin{Highlighting}[]
\FunctionTok{library}\NormalTok{(ggplot2)}
\end{Highlighting}
\end{Shaded}

\begin{verbatim}
## Warning: package 'ggplot2' was built under R version 4.3.2
\end{verbatim}

\begin{Shaded}
\begin{Highlighting}[]
\CommentTok{\# Plotting the top events impacting health}
\FunctionTok{ggplot}\NormalTok{(top\_health\_events, }\FunctionTok{aes}\NormalTok{(}\AttributeTok{x =} \FunctionTok{reorder}\NormalTok{(EVTYPE, }\SpecialCharTok{{-}}\NormalTok{(total\_fatalities }\SpecialCharTok{+}\NormalTok{ total\_injuries)), }
                              \AttributeTok{y =}\NormalTok{ total\_fatalities }\SpecialCharTok{+}\NormalTok{ total\_injuries)) }\SpecialCharTok{+}
  \FunctionTok{geom\_bar}\NormalTok{(}\AttributeTok{stat =} \StringTok{"identity"}\NormalTok{, }\AttributeTok{fill =} \StringTok{"darkred"}\NormalTok{) }\SpecialCharTok{+}
  \FunctionTok{labs}\NormalTok{(}\AttributeTok{title =} \StringTok{"Top 10 Events Impacting Population Health"}\NormalTok{, }
       \AttributeTok{x =} \StringTok{"Event Type"}\NormalTok{, }\AttributeTok{y =} \StringTok{"Number of Fatalities + Injuries"}\NormalTok{) }\SpecialCharTok{+}
  \FunctionTok{theme}\NormalTok{(}\AttributeTok{axis.text.x =} \FunctionTok{element\_text}\NormalTok{(}\AttributeTok{angle =} \DecValTok{45}\NormalTok{, }\AttributeTok{hjust =} \DecValTok{1}\NormalTok{))}
\end{Highlighting}
\end{Shaded}

\includegraphics{Storm-Data-Analysis-Report_files/figure-latex/unnamed-chunk-5-1.pdf}

\hypertarget{economic-impacts-of-weather-events}{%
\subsubsection{Economic Impacts of Weather
Events}\label{economic-impacts-of-weather-events}}

The figure below shows the top 10 severe weather events that cause the
highest economic damage in terms of property and crop damage. Tornadoes,
thunderstorm winds and floods are responsible for the most economic
damage.

\begin{Shaded}
\begin{Highlighting}[]
\CommentTok{\# Plotting the top events with economic impact}
\FunctionTok{ggplot}\NormalTok{(top\_economic\_events, }\FunctionTok{aes}\NormalTok{(}\AttributeTok{x =} \FunctionTok{reorder}\NormalTok{(EVTYPE, }\SpecialCharTok{{-}}\NormalTok{total\_economic\_dmg), }
                                \AttributeTok{y =}\NormalTok{ total\_economic\_dmg)) }\SpecialCharTok{+}
  \FunctionTok{geom\_bar}\NormalTok{(}\AttributeTok{stat =} \StringTok{"identity"}\NormalTok{, }\AttributeTok{fill =} \StringTok{"darkblue"}\NormalTok{) }\SpecialCharTok{+}
  \FunctionTok{labs}\NormalTok{(}\AttributeTok{title =} \StringTok{"Top 10 Events with Greatest Economic Impact"}\NormalTok{, }
       \AttributeTok{x =} \StringTok{"Event Type"}\NormalTok{, }\AttributeTok{y =} \StringTok{"Total Property + Crop Damage (in USD)"}\NormalTok{) }\SpecialCharTok{+}
  \FunctionTok{theme}\NormalTok{(}\AttributeTok{axis.text.x =} \FunctionTok{element\_text}\NormalTok{(}\AttributeTok{angle =} \DecValTok{45}\NormalTok{, }\AttributeTok{hjust =} \DecValTok{1}\NormalTok{))}
\end{Highlighting}
\end{Shaded}

\includegraphics{Storm-Data-Analysis-Report_files/figure-latex/unnamed-chunk-6-1.pdf}

\hypertarget{conclusion}{%
\subsubsection{Conclusion}\label{conclusion}}

Health Impacts: Tornadoes have by far the most devastating impact on
population health, with the highest number of fatalities and injuries
over the 61-year period. Other harmful events include excessive heat,
and Thunderstorm winds.

Economic Impacts: Tornadoes, Thunderstorm winds, Floods, contribute to
the greatest economic damage, with Tornadoes leading in terms of total
property and crop damage.

\end{document}
